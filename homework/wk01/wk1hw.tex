\documentclass[12pt]{article}
 \usepackage[margin=1in]{geometry}
\usepackage{amsmath,amsthm,amssymb,amsfonts}

\newcommand{\N}{\mathbb{N}}
\newcommand{\Z}{\mathbb{Z}}

\newenvironment{problem}[2][Problem]{\begin{trivlist}
\item[\hskip \labelsep {\bfseries #1}\hskip \labelsep {\bfseries #2.}]}{\end{trivlist}}

\newenvironment{solution}{\paragraph{Solution:}}{\hfill}


\usepackage{pgfplots}
\usetikzlibrary{arrows}
\usetikzlibrary{decorations.markings}
\usetikzlibrary{datavisualization}
\usetikzlibrary{datavisualization.formats.functions}
%\usepackage{pstricks-add}

\pgfplotsset{every axis/.append style={
                    axis x line=middle,    % put the x axis in the middle
                    axis y line=middle,    % put the y axis in the middle
                    axis line style={<->,color=gray}, % arrows on the axis
                    xlabel={$x$},          % default put x on x-axis
                    ylabel={$y$},          % default put y on y-axis
            }}
\pgfplotsset{compat=1.15}

\usepackage[parfill]{parskip}
\usepackage[shortlabels]{enumitem}


\newcommand{\pgraph}[4]{
    \begin{center}
    	
    \begin{tikzpicture}
    \begin{axis}[
        trig format plots=rad,
        axis equal,
        grid=both
    ]
    \addplot [domain=#3:#4, variable=\t, samples=50, black, decoration={
        markings,
        mark=between positions 0.2 and 1 step 4em with {\arrow [scale=1.5]{stealth}}
        }, postaction=decorate] 
    ({#1}, {#2});
    
    \end{axis}
    \end{tikzpicture}
    
    \end{center}
}


\begin{document}

%\renewcommand{\qedsymbol}{\filledbox}
%Good resources for looking up how to do stuff:
%Binary operators: http://www.access2science.com/latex/Binary.html
%General help: http://en.wikibooks.org/wiki/LaTeX/Mathematics
%Or just google stuff

\title{Math 53: Week 1 Homework}
\author{Abhijay Bhatnagar}
\maketitle

\setcounter{secnumdepth}{0} %% no numbering
\section{Assignment}

Section 10.1: 1, 2, 3, 5, 7, 11, 12, 24, 37, 38

Section 10.2: 1, 2, 3, 4, 5, 11, 13, 17, 18, 19, 29, 30, 32, 33, 41, 42, 43, 44, 48, 51, 52, 53, 73, 74 

%========Problem 10.1.1=========
\def \xt {1 - t^2}
\def \yt {2*t - t^2}
\def \dl {-1}
\def \dr {2}

\begin{problem}{10.1.1} Graph the parametric equations within the domain.

\[ 	x = \xt \]
\[	y = \yt \]
\[	\dl \leqslant t \leqslant \dr \]
\end{problem}

\begin{solution}


\pgraph{\xt}{\yt}{\dl}{\dr} 

\end{solution}

%========Problem 10.1.2=========

\newpage
\begin{problem}{10.1.2} Graph the parametric equations within the domain.
\[ 	x=t^3-t \]
\[	y = t^2 + 2 \]
\[	-2 \leqslant t \leqslant 2 \]
\end{problem}

\begin{solution}


\pgraph{t^3-t}{t^2 + 2}{-2}{2} 

\end{solution}

%========Problem 10.1.3=========

\begin{problem}{10.1.3} Graph the parametric equations within the domain.
\[ 	x = t + \sin(t) \]
\[	y = \cos(t) \]
\[	-\pi \leqslant t \leqslant \pi \]
\end{problem}

\begin{solution}


\pgraph{t + sin(t)}{cos(t)}{-pi}{pi}

\end{solution}

%========Problem 10.1.5=========
\def \xt {2*t-1}
\def \yt {1/2 * t +1}
\def \dl {-5}
\def \dr {5}

\begin{problem}{10.1.5}

\begin{enumerate}[(a)]
\item Sketch the curve by using the parametric equations to plot points. Indicate with an arrow the direction in which the curve is traced as $t$ increases.
\item Eliminate the parameter to find a Cartesian equation of the curve.
\end{enumerate}

\[ 	x = \xt \]
\[	y = \yt \]
\end{problem}

\begin{solution}

\begin{enumerate}[(a)]
\item
\hskip \labelsep
\pgraph{\xt}{\yt}{\dl}{\dr}

\item

\begin{equation} \label{eq1}
    x = \xt
\end{equation}
    Solving $x$ for $t$ \dots
\begin{equation} \label{eq2}
    t = \frac{x+1}{2}
\end{equation}
    Inserting \eqref{eq2} into the equation for $y$ \dots
\begin{equation}
\begin{split}
    y &=\frac{1}{2}*\frac{x+1}{2}+1 \\
      &=\frac{1}{4} x+\frac{5}{4}
\end{split}
\end{equation}

\end{enumerate}

\end{solution}


%========Problem 10.1.7=========
\newpage
\def \xt {t^2-3}
\def \yt {t + 2}
\def \dl {-3}
\def \dr {3}

\begin{problem}{10.1.7}

\begin{enumerate}[(a)]
\item Sketch the curve by using the parametric equations to plot points. Indicate with an arrow the direction in which the curve is traced as $t$ increases.
\item Eliminate the parameter to find a Cartesian equation of the curve.
\end{enumerate}

\[ 	x = \xt \]
\[	y = \yt \]
\[	\dl \leqslant t \leqslant \dr \]
\end{problem}

\begin{solution}

\begin{enumerate}[(a)]
\item
\hskip \labelsep
\pgraph{\xt}{\yt}{\dl}{\dr}

\item

\begin{equation} \label{eq1}
    x=\xt
\end{equation}
    Solving $x$ for $t$ \dots
\begin{equation} \label{eq2}
    t = \sqrt{x+3}
\end{equation}
    Inserting \eqref{eq2} into the equation for $y$ \dots
\begin{equation}
    y =\sqrt{x+3}+2
\end{equation}

\end{enumerate}

\end{solution}



%========Problem 10.1.11=========

\newpage
\def \xt {sin(1/2*t)}
\def \yt {cos(1/2*t)}
\def \dl {-pi}
\def \dr {pi}

\begin{problem}{10.1.11}

\begin{enumerate}[(a)]
\item Eliminate the parameter to find a Cartesian equation of the curve.
\item Sketch the curve by using the parametric equations to plot points. Indicate with an arrow the direction in which the curve is traced as $t$ increases.
\end{enumerate}

\[ 	x = \xt \]
\[	y = \yt \]
\[	\dl \leqslant t \leqslant \dr \]
\end{problem}

\begin{solution}

\begin{enumerate}[(a)]
\item

\begin{equation} \label{eq1}
    x=\xt
\end{equation}
    Solving $x$ for $t$ \dots
\begin{equation} \label{eq2}
	\def \t{ \frac{\arcsin{x}}{2} }
    t = \t
\end{equation}
    Inserting \eqref{eq2} into the equation for $y$ \dots
\begin{equation}
    y = cos(1/2 * \t)
\end{equation}

\item
\hskip \labelsep
\pgraph{\xt}{\yt}{\dl}{\dr}


\end{enumerate}

\end{solution}



%========Problem 10.1.12=========

%========Problem 10.1.24=========

%========Problem 10.1.37=========
%========Problem 10.1.38=========


%========Problem 10.2.1=========
%========Problem 10.2.2=========
%========Problem 10.2.3=========
%========Problem 10.2.4=========
%========Problem 10.2.5=========









%
%
%
%\begin{problem}{10.1.2}
%x−t3 1t, y−t2 12, 22<t<2
%\end{problem}
%
%\begin{proof}
%Proof goes here. Repeat as needed
%\end{proof}
%
%
%
%\begin{problem}{10.1.3}
%x−t1sint, y−cost, 2 <t< 
%\end{problem}
%
%\begin{proof}
%Proof goes here. Repeat as needed
%\end{proof}

%
%
%\begin{problem}{x.yz}
%Statement of problem goes here
%\end{problem}
%
%\begin{proof}
%Proof goes here. Repeat as needed
%\end{proof}


\end{document}